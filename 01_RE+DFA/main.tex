\documentclass[a4paper, twoside, 12pt]{article}
\usepackage[spanish]{babel}
\usepackage[utf8x]{inputenc}
\usepackage{lmodern}

\usepackage{amsmath}
\usepackage{amsfonts}

\usepackage[pdftex,breaklinks]{hyperref}
\usepackage{xcolor}

\hypersetup{
    colorlinks=true,
    linkcolor=magenta,
    filecolor=black,      
    urlcolor=cyan,
    citecolor=teal,
    pdfauthor = {Joaquín Badillo},
    linktocpage = true
}

\usepackage{geometry}

% Tikz
\usepackage{tikz}
\usetikzlibrary{scopes}
\tikzset{
    vertical align/.style={
    baseline=-.5*(height("$+$")-depth("$+$"))
   }
}

\usetikzlibrary{automata, positioning, arrows}
\tikzset{
    ->,
    node distance=3cm,
    every state/.style={thick, fill=gray!10},
    initial text=$\epsilon$, 
}

\usepackage{fancyhdr}
\pagestyle{fancy}
\fancyhf{}
\fancyhead[OL]{\color{magenta}{\textbf{\thepage}} \quad \tikz[vertical align] \draw[fill = magenta] (0,0) circle (1.25pt); \quad \color{black}{\textsc{Compiladores}}}
\fancyhead[ER]{\color{black}{\textsc{TC3002B}} \quad \color{magenta}{\tikz[vertical align] \draw[fill = magenta] (0,0) circle (1.25pt); \quad \textbf{\thepage}}}
\renewcommand{\headrulewidth}{1pt}
\setlength{\headheight}{15pt}

% Set document properties
\geometry{margin=2.4cm}
\oddsidemargin 0pt
\evensidemargin 0pt
\marginparsep 0pt
\linespread{1.5}

\usepackage{apacite}
\usepackage{textgreek}

\raggedright
\usepackage[skip=6pt]{parskip}

\begin{document}
\begin{titlepage}
\begin{center}
		\includegraphics[width=0.75\textwidth]{Img/logo.png}
		
		\vspace{20pt}
		
		\begin{LARGE}\bf{Tarea 1}
		\end{LARGE}
		
		\vspace{50pt}

		TC3002B - Grupo 501
		
		\vspace{80pt}
        
        \textbf{Joaquín Badillo Granillo}
        
        Tecnológico de Monterrey
        
        Campus Santa Fe
        
        \href{mailto:a01026364@tec.mx}{a01026364@tec.mx}            
		
        \vspace{120pt}     
    				
		2 de Abril de 2025
	\end{center}
\end{titlepage}
\pagebreak
\section*{Tarea 1}

\subsection{Descríbanse con palabras los lenguajes representados por las siguientes expresiones regulares}

\begin{enumerate}
    \item \texttt{0(0|1)*0}
    
    Todas las cadenas de dígitos binarios que empiezan con \texttt{0} y terminan con \texttt{0} (contienen una cantidad arbitraria de \texttt{1} o \texttt{0}s intermedios).
    
    \item \texttt{(a|b)*a(a|b|\textepsilon)}
    
    Todas las cadenas de texto en el alfabeto \textSigma = \{\texttt{a,b}\} cuya última o penúltima letra (o ambas) es la \texttt{a}.
    
    \item \texttt{((\textepsilon|0)1*)*}

    Cualquier cadena de dígitos binarios.
    
    \item \texttt{(A|B|...|Z)(a|b|...|z)*}

    Las cadenas de letras que solamente tienen una mayúscula al inicio y tienen una cantidad arbitraria de letras minúsculas (o ninguna minúscula) después.
    
    \item \texttt{(0|1)*0(0|1)(0|1)}

    Cadenas de dígitos binarios donde el antepenúltimo dígito es el \texttt{0}.
\end{enumerate}


\subsection{Escriba expresiones regulares para los siguientes lenguajes o indique si no se pueden escribir expresiones regulares para ellos}

\begin{enumerate}
    \item Todas las cadenas de letras minúsculas que inicien y terminen con una \texttt{a}.

    \texttt{a([a-z]*a)?}
    
    \item Todas las cadenas de dígitos que no contengan ceros al inicio.

    \texttt{[1-9][0-9]*}
    
    \item Todas las cadenas de dígitos que representen números pares.

    \texttt{[0-9]*[02468]}
    
    \item Comentarios que consisten en una cadena encerrada entre \texttt{/*} y \texttt{*/}, sin ningún \texttt{*/} intermedio a menos que aparezca entre comillas simples \texttt{‘ ‘}.

    \texttt{{/\color{magenta}{\textbackslash*}}(('[\textasciicircum']*')|[\textasciicircum']*)*{\color{magenta}{\textbackslash*}}/}
    
    \item Un número en C++, sea entero o flotante, y si es flotante, permite usar la notación exponencial con \texttt{E} y un exponente \texttt{+} o \texttt{-}, o nada después del punto.

    \texttt{[+{\color{magenta}{\textbackslash-}}]?[0-9]*({\color{magenta}{\textbackslash.}}[0-9]*([eE][+{\color{magenta}{\textbackslash-}}]?[0-9]+)?)?}
\end{enumerate}


\subsection{Dibuje un DFA para cada uno de los conjuntos de caracteres}
\begin{enumerate}

    \item Todas las cadenas de letras minúsculas que inicien y terminen con una \texttt{a}.

    \begin{figure}[ht]
        \centering
        \begin{tikzpicture}
        \node[state, initial] (q0) {Inicio};
        \node[state, accepting, right of=q0] (q1) {$q_1$};
        \node[state, right of=q1] (q2) {$q_2$};
        \draw
        (q0) edge[above] node{a} (q1)
        (q1) edge[loop above] node{a} (q1)
        (q1) edge[bend left, above] node{[b-z]} (q2)
        (q2) edge[bend left, below] node{a} (q1);
        \end{tikzpicture}
    \end{figure}

    \item Todas las cadenas de letras minúsculas que inicien o terminen con una \texttt{a} (o ambos). Observe que si no inicia con \texttt{a} debe terminar forzosamente con \texttt{a}.

    \begin{figure}[ht]
        \centering
        \begin{tikzpicture}
        \node[state, initial] (q0) {Inicio};
        \node[state, accepting, right of=q0] (q1) {$q_1$};
        \node[state, below of=q1] (q2) {$q2$};
        \node[state, accepting, right of=q2] (q3) {$q3$};
        \draw
        (q0) edge[bend left, above] node{a} (q1)
        (q1) edge[loop above] node{[a-z]} (q1)
        (q0) edge[bend right, below] node{[b-z]} (q2)
        (q2) edge[loop above] node{[b-z]} (q2)
        (q2) edge[bend left, above] node{a} (q3)
        (q3) edge[loop above] node{a} (q3)
        (q3) edge[bend left, below] node{[b-z]} (q2);
        \end{tikzpicture}
    \end{figure}

    \pagebreak
    
    \item Todas las cadenas de dígitos que no inicien con \texttt{0}.

    \begin{figure}[ht]
        \centering
        \begin{tikzpicture}
        \node[state, initial] (q0) {Inicio};
        \node[state, accepting, right of=q0] (q1) {$q_1$};
        \draw
        (q0) edge[above] node{[1-9]} (q1)
        (q1) edge[loop above] node{[0-9]} (q1);
        \end{tikzpicture}
    \end{figure}

    \item Todas las cadenas de dígitos que representen números pares.

    \begin{figure}[ht]
        \centering
        \begin{tikzpicture}
        \node[state, initial] (q0) {Inicio};
        \node[state, right of=q0] (q1) {$q_1$};
        \node[state, accepting, below of=q1] (q2) {$q_2$};
        \draw
        (q0) edge[bend left, above] node{[13579]} (q1)
        (q0) edge[bend right, left] node{[02468]} (q2)
        (q2) edge[loop right, right] node{[02468]} (q2)
        (q2) edge[bend left, above left] node{[13579]} (q1)
        (q1) edge[bend left, right] node{[24680]} (q2);
        \end{tikzpicture}
    \end{figure}

    \item Todas las cadenas de \texttt{a} y \texttt{b} que no contengan tres \texttt{b}.

    \begin{figure}[ht]
        \centering
        \begin{tikzpicture}
        \node[state, initial] (q0) {Inicio};
        \node[state, accepting, right of=q0] (b0) {$b_0$};
        \node[state, accepting, below of=b0] (b1) {$b_1$};
        \node[state, accepting, right of=b1] (b2) {$b_2$};
        \node[state, right of=b2] (b3) {$b_3$};
        \draw
        (q0) edge[left, above] node{a} (b0)
        (b0) edge[loop, above] node{a} (b0)
        (q0) edge[bend right, left] node{b} (b1)
        (b0) edge[right, right] node{b} (b1)
        (b1) edge[loop below, below] node{a} (b1)
        (b1) edge[right, above] node{b} (b2)
        (b2) edge[right, above] node{b} (b3)
        (b2) edge[loop, above] node{a} (b2)
        (b3) edge[loop, above] node{[ab]} (b3);
        \end{tikzpicture}
    \end{figure}

    \pagebreak
    
    \item Todas las cadenas de \texttt{0} y \texttt{1} que contengan un número par de \texttt{0} y un número impar de \texttt{1}.

    \begin{figure}[ht]
        \centering
        \begin{tikzpicture}
        \node[state, initial] (ff) {$0_{\overline{2\mathbb{Z}}},1_{2\mathbb{Z}}$};
        \node[state, below of=ff] (tf) {$0_{2\mathbb{Z}},1_{2\mathbb{Z}}$};
        \node[state, accepting, right of=tf] (tt) {$0_{2\mathbb{Z}},1_{\overline{2\mathbb{Z}}}$};
        \node[state, above of=tt] (ft) {$0_{\overline{2\mathbb{Z}}},1_{\overline{2\mathbb{Z}}}$};
        \draw
        (tt)edge[bend left, left] node{0}(ft)
        (tt)edge[bend right, above] node{1}(tf)
        (tf)edge[bend right, below] node{1}(tt)
        (tf)edge[bend right, right] node{0}(ff)
        (ff)edge[bend right, left] node{0}(tf)
        (ff)edge[bend right, above] node{1}(ft)
        (ft)edge[bend left, right] node{0}(tt)
        (ft)edge[bend right, above] node{1}(ff);
        \end{tikzpicture}
    \end{figure}
\end{enumerate}
\end{document}